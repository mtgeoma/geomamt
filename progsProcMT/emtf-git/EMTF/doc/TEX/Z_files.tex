\documentstyle[12pt]{article}
\setlength{\textheight}{23cm}
\setlength{\textwidth}{16cm}
\setlength{\oddsidemargin}{0.25in}
\setlength{\topmargin}{-0.5in}
%\setlength{\parskip}{.5in}
\title{Errors Bars for Transfer Function Elements in Z--files}
\author{Gary D. Egbert\thanks{College of Oceanic and Atmospheric Sciences, Oregon State University, Corvallis; egbert@oce.orst.edu}}

\begin{document}

\maketitle

\section{File Format}

The Z--files contain all information needed to compute standard
transfer functions (e.g., impedances), with error bars in any coordinate
system.  Here is an overview of the format, followed by an
artificial example, with some annotation off to the side.
The basic idea is that there are {\it NCH} channels, with the first
two used as the ``local reference''i.e., these are the input
or predictor channels (classically the local horizontal  magnetics), and
the remaining {\it NCH-2} are the output or predicted channels (the electrics,
and/or vertical magnetics).  Note that there might also have been
another pair of channels (or a whole array) used as a remote reference.
These possible other channels are not referred to explicitly in this
file (but they were used to compute the contents of the file).  A
file of this same format can in principle be produced from single
station, standard remote reference, or the multiple station program.
This effects how the contents of this file was created, but not
any subsequent calculations using this file.

In overview, the file is ASCII, with a short header block
which identifies the {\it NCH} channels.
There are then a series of {\it NBANDS} blocks,
one for each period for which an estimate has been computed.
Each period block contains three complex arrays:

1) {\bf Z} : the transfer function (TF) array.  This array is {\it NCH-2} rows
by 2 columns.  For {\it NCH = 4},
with two reference channels $H_x$ and $H_y$,
and two predicted channels $E_x$ and $E_y$, {\bf Z} is just the impedance
tensor.

2) {\bf S} : the ``inverse signal covariance'' array.  This is a $2 \times 2$
Hermitian matrix.  Only the 3 elements corresponding to the part
on and below the diagonal are actually in the file.  These elements
are given in the order $S_{11}$, $S_{21}$, $S_{22}$.  The missing element
satisfies $S_{12}$ = ${S_{21}}^*$, where the superscript asterisk denotes
the complex conjugate.  Note that in the case of a single station
impedance estimate {\bf S} is just the inverse of the ${\bf H}$ cross power
matrix.  The exact form is slightly different for the case of remote reference
or array results.  This matrix is needed for the error calculation.
(Actually only the diagonal elements, which are real,
are needed unless you rotate the coordinate systems).

3) {\bf N} the residual covariance matrix.  
This is an $(NCH-2) \times (NCH-2)$ Hermitian matrix,
output in the same symmetric form as {\bf S}.  This
gives the covariance of the residuals
for all predicted channels.  Again,
only the diagonals (also real) of this matrix are needed for 
error calculations in the ``default'' coordinate system, but 
other parts of the matrix will be used for a correct treatment of 
coordinate changes/rotations.


(Mike: Actually there might be a slight difference
in the header block format in the version you have; this is
for the most recent version, and I'm not sure exactly what Clark is
using).

\small
\begin{verbatim}
 TRANSFER FUNCTIONS IN MEASUREMENT COORDINATES    <====  line 1 of file
 ********** WITH FULL ERROR COVARAINCE*********
                                                                                
S2                                             <===== some sort of "station" id
coordinate    49.28   102.91 declination    0.00    <=== station coordinates
number of channels   5   number of frequencies  26  <===  NCH, NBANDS
 orientations and tilts of each channel 
    5     0.00     0.00 S2H  Hx     <===  for each channel in this "station":
    6    90.00     0.00 S2H  Hy     (1) channel number (this came from multmtrn
    7     0.00     0.00 S2H  Hz         and so the #s aren't 1,2 ... ;
    8     0.00     0.00 S2E  Ex      (2) orientation (deg. E of N);  (3) tilt ;
    9    90.00     0.00 S2E  Ey      (4) Data loger ID  (5) channel type
 
period :      4.65455    decimation level   1    freq. band from   25 to   30
number of data point   2496 sampling freq.   1.000 Hz   <=== info about 1 band
 Transfer Functions
  0.2498E+00 -0.2049E-03 -0.9341E-04  0.2517E+00
 -0.6246E-02 -0.5245E-01 -0.7291E+01 -0.7318E+01       <=== Z
  0.7292E+01  0.7346E+01 -0.3806E-01  0.5754E-02
 Inverse Coherent Signal Power Matrix
  0.2947E-07  0.5753E-16                               <=== S
 -0.1575E-09  0.1391E-09  0.2895E-07  0.2386E-15
 Residual Covariance
  0.3198E+02  0.0000E+00
  0.2252E+03 -0.2185E+03  0.2660E+05  0.0000E+00        <=== N
  0.2424E+03  0.2418E+03  0.4577E+03  0.3710E+03  0.2781E+05  0.0000E+00
\end{verbatim}
\normalsize

The block given above for one period should be pretty much self explanatory.
Note that the last two rows of the TF matrix are the
local impedance tensor.

\section{Error Calculation}

First I just give the formula for calculating errors in the
TF given in the files (i.e., in the ``default''
measurement coordinate system).  Next, I'll give formulas for
transforming the matrices {\bf Z, S, N} into a different
coordinate system (not necessarily by rotation).  The initial formulas
for TF error in the measurement coordinate system
can then be applied to the transformed 
matrices.  Finally, linear error propagation is applied to give
the standard error estimates for $\rho_a$ and $\phi$ computed
from the off-diagonal elements of the impedance.

\subsection{Errors In Transfer Functions}

The error covariance for the elements of the transfer
function matrix {\bf Z} is given by:
\begin{equation}
{\bf Cov} [ Z_{ij} Z_{i'j'} ] = N_{ii'} S_{jj'} \,\,\,\,\,\,\, j, j' = 1,2 
\,\,\,\,\,\,,\  i, i' = 1,NCH-2   .
\label{eq1}
\end{equation}
You will normally only care about the variances (i.e., the case where
$i = i'$ and $j = j'$).  In this case you would use only the
diagonal elements of {\bf S} and {\bf N}.  In the following I
refer to these variances as
\begin{equation}
{\sigma_{ij}}^2 = 
{\bf Var} [ Z_{ij} ] = {\bf Cov} [ Z_{ij} Z_{ij} ] = 
N_{ii} S_{jj} .
\label{eq2}
\end{equation}
For example, the impedance element $Z_{xy} ( = E_x / H_y )$
in the above example is element (2,2) (row = i = 2, column = j = 2)
in the TF matrix {\bf Z}.  The error variance is obtained from the product
of the second diagonal element of the inverse coherent signal
power matrix $S_{22}$, and the second diagonal element of the residual
covariance $N_{22}$.  The other off-diagonal impedance element $Z_{yx}$
corresponds to $i = 3$ and $j = 1$, and the variance 
is $\sigma_{31} = N_{33} S_{11}$.
Note that this gives the variances of the complex transfer
functions; variances of real and
imaginary parts separately are each one half of the complex variance
given by (\ref{eq1}).

\subsection{Transformation Of Transfer Functions and Errors}

The transformation of error covariance can be computed for
any linear transformation of the predicted and predictor
channels.  Here I just give expressions
for the most standard rotations.  Denote by $\theta_1 , \theta_2 , ...
\theta_{NCH}$ the channel orientations (these are given in the header
block of the Z\_ file).
Let $\theta$ be the desired rotation of the $x-$axis, relative to
the same reference direction used to define the channel orientations
(e.g., geographic or geomagnetic north).  Note that the sort of coordinate
changes we focus on here implicitly involve pairs of channels (the two
reference magnetics; a pair of electric channels).  Vertical magnetics
are not rotated (well ... we {\it could} get into allowing for tilt ...),
and when there are multiple electrics, it will be necessary to identify
pairs of channels to transform together.  I thus describe transformation
of one pair of channels at a time, say channels $l, m$.  Form the matrix
\begin{equation}
{\bf U_{lm}}  =
\left [
\matrix { \cos ( \theta_l - \theta ) & \sin ( \theta_l - \theta ) \cr
          \cos ( \theta_l - \theta ) & \sin ( \theta_m - \theta ) \cr}
\right ] ^{-1}
\label{rotmat}
\end{equation}

Note that if you form the 2-vector ${\bf x }$ from the $(l, m)$
pair of measured data channels, then ${\bf U_{lm}} { \bf x}$ gives the
vector expressed in the new right-handed orthogonal coordinate system
(with $x-$axis pointing
in the direction $\theta$ degrees E of the reference direction).
Note that in the ``usual'' MT case where
there is one pair of reference channels $H_x$, $H_y$ and one pair
of predicted channels $E_x$, $E_y$, and both are expressed in
the same orthogonal coordinate system, then the same matrix ${\bf U_{lm}}$
would be used for coordinate transformation of
both pairs, and we would also have
$\theta_m = \theta_l + 90$.  In this case ${\bf U_{lm}}$ would
reduce to the more familiar form for the impedance tensor
rotation matrix.  The formulas given here work for any orientations,
including the case of non-orthogonal measurment component pairs.

First consider transformation of the predicting channels $l = 1$,
$m = 2$ (normally these would be $H_x$, $H_y$).
${\bf Z}$ and ${\bf S}$ are effected
by this part of the transformation.  In the new coordinate system the
matrices are:
\begin{equation}
{\bf Z'} = {\bf Z U_{12}}^T    \,\,\,\,\,\,\,\,\,\,\,\,
{\bf S'} = {\bf U_{12} S U_{12}}^T
\label{hxf}
\end{equation}
The output residual covariance ${\bf N}$ of course remains unchanged
by a transformation of only the input channels. 

Next consider transformation of two 
of the output channels, $3 \le l, m \le NCH$.
(Note the numbering convention: output channels start with 3, and go to $NCH$,
for a total of $NCH-2$).  The simplest way to express the result in
general is to define an $(NCH-2) \times (NCH-2)$ 
transformation matrix ${\bf V_{lm}}$ which rotates only channels $l$ and $m$.
For the example file above, where $NCH = 5$,
the matrix for
rotating the coordinate system for the pair of electric field channels
(i.e., $l = 4, m = 5$), ${\bf V_{45}}$ would take the form
\begin{equation}
{\bf V_{45}}  =
\left [
\matrix {   1 & 0 & 0 \cr
         0 & \cos ( \theta_4 - \theta ) & \cos ( \theta_5 - \theta ) \cr
         0 & \sin ( \theta_4 - \theta ) & \sin ( \theta_5 - \theta ) \cr}
\right]  .
\label{vdef}
\end{equation}
More generally the following pseudo-code defines ${\bf V_{lm}}$,
assuming $l < m$:

\begin{list} {} {\leftmargin=2em}
\item ${\bf V_{lm}} = (NCH-2) \times (NCH-2)$ identity  matrix
\item ${\bf V_{lm}}(l-2,l-2) =  \cos ( \theta_l - \theta )$
\item ${\bf V_{lm}}(m-2,l-2) =  \sin ( \theta_l - \theta )$
\item ${\bf V_{lm}}(l-2,m-2) =  \cos ( \theta_m - \theta )$
\item ${\bf V_{lm}}(m-2,m-2) =  \sin ( \theta_m - \theta )$
\end{list}
With ${\bf V_{lm}}$ thus defined the transformations of {\bf Z} and {\bf N} are:
\begin{equation}
{\bf Z'} = {\bf V_{lm} Z }   \,\,\,\,\,\,\,\,\,\,\,\,
{\bf N'} = {\bf V_{lm} N V_{lm} }^T
\label{rxf}
\end{equation}
In general both input and output channels will be rotated, so
both (\ref{hxf}) and (\ref{rxf}) will be used.  In the 5 channel
example given above the full transfromation of all arrays is thus:
\begin{equation}
{\bf Z'} = {\bf V_{45} Z U_{12} } \sp {T}   \,\,\,\,\,\,\,\,\,\,\,\,
{\bf S'} = {\bf U_{12} S U_{12} } \sp {T}   \,\,\,\,\,\,\,\,\,\,\,\,
{\bf N'} = {\bf V_{45} N V_{45} }^T .
\label{hexf}
\end{equation}
More generally
there may be a series of electric field pairs, requiring that
(\ref{rxf}) be applied for each pair.  Note that in this case
a single matrix {\bf V} can be derived which transforms all
channel pairs, by starting with the $(NCH-2) \times (NCH-2)$
identity matrix and modifying the
appropriate four elements of {\bf V} for each pair $l, m$.
Error (co)variances for the transformed impedance elements are then as
given in (\ref{eq2}) and (\ref{eq1}), with ${\bf N'}$ and ${\bf S'}$
replacing {\bf N} and {\bf S}.

\section{Apparent Resistivities and Phases}

After transforming all three arrays,
apparent resistivities ($\rho_a$), phases ($\phi$) and error
bars ($\sigma_{\rho}; \sigma_{\phi}$) can be computed from 
the appropriate off-diagonal impedance
elements (say $Z_{ij}$), the period $T$,
and the associated error variance $\sigma_{ij}$ given above.
For completness here are the expressions derived from linear
propogation of errors, under the assumption that errors are small
compared to the impedance.

$$
\rho_a = { { T | Z_{ij} |^2 } \over {5} }
$$

$$
\sigma_{\rho} = \left [ {{ ( 2T \rho {\sigma_{ij}}^2 } \over 5 } \right ]
\sp {1/2}
$$

$$
\phi = { 180 \over \pi }  \arctan [ \Im ( Z_{ij} ) / \Re ( Z_{ij} ) ] 
$$

$$
\sigma_{\phi}  = { 180 \over { \pi | Z_{ij} | } }
\left [ { {\sigma_{ij}}^2 \over 2 }
\right ] \sp {1/2}
$$

\end{document}

